
The following process can be used to fit parameters to an eversion data record where the record consists of
the following data as a function of time:

\begin{itemize}
    \item Pressure in the tube storage chamber (Pa)
    \item Air flow from source into tube storage chamber ($m^3/sec$)
    \item Length of everting tube relative to chamber opening (m)
    \item Velocity of eversion (derived) ($m/sec$)
\end{itemize}

These can be plotted multiple ways such as in Fig \ref{xxxxxx} for example.


\begin{itemize}
    \item  Adjust load line pressure intercept.

    {\bf Data Focus: } Pressure/Flow curves (upper left):\\
    adjust Psource_SIu to move the load line and sim trajectory (they should overlap)
    up and down.  Adjust Rsource_SIu to adjust its slope (higher values slope down more).

    \item Adjust stop-start thresholds.

     {\bf Data Focus: } Pressure-Time curves (middle left):\\
    Adjust PBA_static up or down to match the pressure peaks in experiment (green dashed).
    Adjust PHalt_dyn  up or down to match the pressure valleys.

    \item Adjust threshold taper.

     {\bf Data Focus: } Pressure-Time curves (middle left):\\
    Adjust Threshold Taper up or down to speed or slow down convergence of the thresholds.

    \item Adjust Friction or drag

     {\bf Data Focus: } Length-Time curves (middle right):\\
    Adjust the viscus drag constant of tubing (K_drag) up or down to match the overall slope
    of the data trace.

    Adjust max tubing length (Lmax) to match stopping point of data.

    \tiem Experiment with other parameters or go back and repeat the procedure.

\end{itemize}
